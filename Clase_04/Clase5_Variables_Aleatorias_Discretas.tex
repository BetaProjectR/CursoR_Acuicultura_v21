<<<<<<< HEAD
\PassOptionsToPackage{unicode=true}{hyperref} % options for packages loaded elsewhere
\PassOptionsToPackage{hyphens}{url}
%
\documentclass[ignorenonframetext,]{beamer}
=======
% Options for packages loaded elsewhere
\PassOptionsToPackage{unicode}{hyperref}
\PassOptionsToPackage{hyphens}{url}
%
\documentclass[
  ignorenonframetext,
]{beamer}
>>>>>>> e735aca38ee9621f00e78475fcc0664f62caa5b7
\usepackage{pgfpages}
\setbeamertemplate{caption}[numbered]
\setbeamertemplate{caption label separator}{: }
\setbeamercolor{caption name}{fg=normal text.fg}
\beamertemplatenavigationsymbolsempty
<<<<<<< HEAD
% Prevent slide breaks in the middle of a paragraph:
\widowpenalties 1 10000
\raggedbottom
\setbeamertemplate{part page}{
\centering
\begin{beamercolorbox}[sep=16pt,center]{part title}
  \usebeamerfont{part title}\insertpart\par
\end{beamercolorbox}
}
\setbeamertemplate{section page}{
\centering
\begin{beamercolorbox}[sep=12pt,center]{part title}
  \usebeamerfont{section title}\insertsection\par
\end{beamercolorbox}
}
\setbeamertemplate{subsection page}{
\centering
\begin{beamercolorbox}[sep=8pt,center]{part title}
  \usebeamerfont{subsection title}\insertsubsection\par
\end{beamercolorbox}
=======
% Prevent slide breaks in the middle of a paragraph
\widowpenalties 1 10000
\raggedbottom
\setbeamertemplate{part page}{
  \centering
  \begin{beamercolorbox}[sep=16pt,center]{part title}
    \usebeamerfont{part title}\insertpart\par
  \end{beamercolorbox}
}
\setbeamertemplate{section page}{
  \centering
  \begin{beamercolorbox}[sep=12pt,center]{part title}
    \usebeamerfont{section title}\insertsection\par
  \end{beamercolorbox}
}
\setbeamertemplate{subsection page}{
  \centering
  \begin{beamercolorbox}[sep=8pt,center]{part title}
    \usebeamerfont{subsection title}\insertsubsection\par
  \end{beamercolorbox}
>>>>>>> e735aca38ee9621f00e78475fcc0664f62caa5b7
}
\AtBeginPart{
  \frame{\partpage}
}
\AtBeginSection{
  \ifbibliography
  \else
    \frame{\sectionpage}
  \fi
}
\AtBeginSubsection{
  \frame{\subsectionpage}
}
\usepackage{lmodern}
\usepackage{amssymb,amsmath}
\usepackage{ifxetex,ifluatex}
<<<<<<< HEAD
\usepackage{fixltx2e} % provides \textsubscript
\ifnum 0\ifxetex 1\fi\ifluatex 1\fi=0 % if pdftex
  \usepackage[T1]{fontenc}
  \usepackage[utf8]{inputenc}
  \usepackage{textcomp} % provides euro and other symbols
\else % if luatex or xelatex
  \usepackage{unicode-math}
  \defaultfontfeatures{Ligatures=TeX,Scale=MatchLowercase}
\fi
% use upquote if available, for straight quotes in verbatim environments
\IfFileExists{upquote.sty}{\usepackage{upquote}}{}
% use microtype if available
\IfFileExists{microtype.sty}{%
\usepackage[]{microtype}
\UseMicrotypeSet[protrusion]{basicmath} % disable protrusion for tt fonts
}{}
\IfFileExists{parskip.sty}{%
\usepackage{parskip}
}{% else
\setlength{\parindent}{0pt}
\setlength{\parskip}{6pt plus 2pt minus 1pt}
}
\usepackage{hyperref}
\hypersetup{
            pdftitle={Clase 5 Variables aleatorias Discretas},
            pdfborder={0 0 0},
            breaklinks=true}
\urlstyle{same}  % don't use monospace font for urls
=======
\ifnum 0\ifxetex 1\fi\ifluatex 1\fi=0 % if pdftex
  \usepackage[T1]{fontenc}
  \usepackage[utf8]{inputenc}
  \usepackage{textcomp} % provide euro and other symbols
\else % if luatex or xetex
  \usepackage{unicode-math}
  \defaultfontfeatures{Scale=MatchLowercase}
  \defaultfontfeatures[\rmfamily]{Ligatures=TeX,Scale=1}
\fi
% Use upquote if available, for straight quotes in verbatim environments
\IfFileExists{upquote.sty}{\usepackage{upquote}}{}
\IfFileExists{microtype.sty}{% use microtype if available
  \usepackage[]{microtype}
  \UseMicrotypeSet[protrusion]{basicmath} % disable protrusion for tt fonts
}{}
\makeatletter
\@ifundefined{KOMAClassName}{% if non-KOMA class
  \IfFileExists{parskip.sty}{%
    \usepackage{parskip}
  }{% else
    \setlength{\parindent}{0pt}
    \setlength{\parskip}{6pt plus 2pt minus 1pt}}
}{% if KOMA class
  \KOMAoptions{parskip=half}}
\makeatother
\usepackage{xcolor}
\IfFileExists{xurl.sty}{\usepackage{xurl}}{} % add URL line breaks if available
\IfFileExists{bookmark.sty}{\usepackage{bookmark}}{\usepackage{hyperref}}
\hypersetup{
  pdftitle={Clase 5 Variables aleatorias Discretas},
  hidelinks,
  pdfcreator={LaTeX via pandoc}}
\urlstyle{same} % disable monospaced font for URLs
>>>>>>> e735aca38ee9621f00e78475fcc0664f62caa5b7
\newif\ifbibliography
\usepackage{graphicx,grffile}
\makeatletter
\def\maxwidth{\ifdim\Gin@nat@width>\linewidth\linewidth\else\Gin@nat@width\fi}
\def\maxheight{\ifdim\Gin@nat@height>\textheight\textheight\else\Gin@nat@height\fi}
\makeatother
% Scale images if necessary, so that they will not overflow the page
% margins by default, and it is still possible to overwrite the defaults
% using explicit options in \includegraphics[width, height, ...]{}
\setkeys{Gin}{width=\maxwidth,height=\maxheight,keepaspectratio}
<<<<<<< HEAD
\setlength{\emergencystretch}{3em}  % prevent overfull lines
\providecommand{\tightlist}{%
  \setlength{\itemsep}{0pt}\setlength{\parskip}{0pt}}
\setcounter{secnumdepth}{0}

% set default figure placement to htbp
\makeatletter
\def\fps@figure{htbp}
\makeatother


\title{Clase 5 Variables aleatorias Discretas}
\providecommand{\subtitle}[1]{}
\subtitle{Diplomado en Análisis de datos con R para la acuicultura}
\author{true}
\date{29 April 2021}
=======
% Set default figure placement to htbp
\makeatletter
\def\fps@figure{htbp}
\makeatother
\setlength{\emergencystretch}{3em} % prevent overfull lines
\providecommand{\tightlist}{%
  \setlength{\itemsep}{0pt}\setlength{\parskip}{0pt}}
\setcounter{secnumdepth}{-\maxdimen} % remove section numbering

\title{Clase 5 Variables aleatorias Discretas}
\subtitle{Diplomado en Análisis de datos con R para la acuicultura}
\author{true}
\date{29 abril 2021}
>>>>>>> e735aca38ee9621f00e78475fcc0664f62caa5b7

\begin{document}
\frame{\titlepage}

\begin{frame}{}
\protect\hypertarget{section}{}

\textbf{PLAN DE CLASE}

\textbf{1).} \textbf{Introducción}

\begin{itemize}
\item
  \textbf{Preguntas al curso.}
\item
  \textbf{Estudio de caso.}
\end{itemize}

\textbf{2).} \textbf{Práctica con R y Rstudio cloud}

\begin{itemize}
\item
  \textbf{Observa y predice el comportamiento de variables aleatorias
  discretas (Binomial y Bernoulli).}
\item
  \textbf{Escribir un código de programación} o \textbf{\emph{script}}
\item
  \textbf{Elaborar un reporte html con Rmarkdown.}
\end{itemize}

\end{frame}

\begin{frame}{}
\protect\hypertarget{section-1}{}

\textbf{Introducción}

\textbf{\emph{Clase 5 -- Variables discretas}}

\end{frame}

\begin{frame}{}
\protect\hypertarget{section-2}{}

\textbf{PREGUNTAS AL CURSO}

<<<<<<< HEAD
\textbf{1).} \textbf{¿ Por qué es importante analizar variables
discretas?}

\textbf{2).} \textbf{¿ Has tenido que analizar este tipo de variables
alguna vez?}
=======
\textbf{1).} \textbf{Señale 3 razones de la importancia de reconocer y
analizar adecuadamente las variables discretas}.

\textbf{2).} \textbf{Señale 3 ejemplos de variables discretas que haya
trabajado o conozca, indique para cada una de ellas que tipo de
distribución tiene la variable}.
>>>>>>> e735aca38ee9621f00e78475fcc0664f62caa5b7

\end{frame}

\begin{frame}{}
\protect\hypertarget{section-3}{}

<<<<<<< HEAD
\textbf{ESTUDIO DE CASO:}
=======
\textbf{ESTUDIO DE CASO}

\textbf{¿Cuál fue la variable en estudio?}

\textbf{¿Qué tipo de distribución usaron?}
>>>>>>> e735aca38ee9621f00e78475fcc0664f62caa5b7

\end{frame}

\begin{frame}{}
\protect\hypertarget{section-4}{}

<<<<<<< HEAD
\textbf{TRABAJO EN GRUPO}

\textbf{¿Cuál fue la variable en estudio?}

\textbf{¿Qué tipo de distribución usaron?}

\textbf{¿Qué software se utilizó para los análisis?}

\begin{block}{}

\textbf{EXPERIMENTO BERNOULLI}

Es un experimento que tiene las siguientes características:

\begin{itemize}
\item
  En cada prueba del experimento sólo son posibles dos resultados éxito
  y fracaso.
\item
  El resultado obtenido en cada prueba es independiente de los
  resultados anteriores.
\item
  La probabilidad de éxito es constante, P(éxito) = p, y no varia de una
  prueba a otra.
\end{itemize}

\begin{block}{}

=======
\textbf{VARIABLE ALEATORIA DISCRETA CON DISTRIBUCIÓN BERNOULLI}

VENENO PARALIZANTE DE LOS MARISCOS (VPM)

Producido en Chile por una microalga llamada Alexandrium catenella.

\includegraphics{~/GitHub/DiplomadoR_Acuicultura_v21/Imagenes/Clase5/LetalidadVPM.png}

Intoxicaciones por VPM en Chile 1972-2002 456 enfermos, 30 fallecidos
(total = 486) Letalidad= 30 / 486 = 0.0617

\begin{block}{}

\textbf{DESCRIBIR EL COMPORTAMIENTO DE MORTALIDAD POR VPM}

Ejemplo: Ocurre un evento de marea roja no detectado con A. catenella,
80 personas resultan intoxicadas en todo Chile.

Si la probabilidad de muerte x VPM es de 0,0617.

¿Cuántas personas morirán?

\includegraphics{~/GitHub/DiplomadoR_Acuicultura_v21/Imagenes/Clase5/LetalidadVPM.png}

\begin{block}{}

\textbf{EXPERIMENTO BERNOULLI}

>>>>>>> e735aca38ee9621f00e78475fcc0664f62caa5b7
\end{block}

\includegraphics{~/GitHub/DiplomadoR_Acuicultura_v21/Imagenes/Clase5/EnsayoBer.png}

\end{block}

\end{frame}

\begin{frame}{}
\protect\hypertarget{section-7}{}

\textbf{DISTRIBUCIÓN BERNOULLI}

\includegraphics{~/GitHub/DiplomadoR_Acuicultura_v21/Imagenes/Clase5/DistBer.png}

\end{frame}

\begin{frame}{}
\protect\hypertarget{section-8}{}

<<<<<<< HEAD
\includegraphics{~/GitHub/DiplomadoR_Acuicultura_v21/Imagenes/Clase5/Bernoulli.png}

\end{frame}

\begin{frame}{}
\protect\hypertarget{section-9}{}

\textbf{FUNCIÓN DE PROBABILIDAD DE UNA VARIABLE ALEATORIA DISCRETA}

Es la función que asocia a cada valor x de la v.a. X su probabilidad p.

*Los valores que toma una v.a. discreta X y sus correspondientes
probabilidades suelen disponerse en una tabla con dos filas o dos
columnas llamada tabla de distribución de probabilidad:

\includegraphics{~/GitHub/DiplomadoR_Acuicultura_v21/Imagenes/Clase5/FP_binomial.png}

Toda función de probabilidad se verifica

\textbf{p1+p2+p3+\ldots{}+pn=1}

\end{frame}

\begin{frame}{}
\protect\hypertarget{section-10}{}

\textbf{FUNCIÓN DE DISTRIBUCIÓN DE UNA VARIABLE ALEATORIA DISCRETA}

Sea X una v.a. cuyos valores suponemos ordenados de menor a mayor.Se
llama función de distribución de la variable X a la función que asocia a
cada valor de la v.a. la probabilidad acumulada hasta ese valor, es
decir,

F(X)= P(X\textless{}=X).
=======
\textbf{HISTOGRAMA Y FUNCIÓN DE DENSIDAD}

\includegraphics{~/GitHub/DiplomadoR_Acuicultura_v21/Imagenes/Clase5/HistBer.png}
>>>>>>> e735aca38ee9621f00e78475fcc0664f62caa5b7

\end{frame}

\begin{frame}{}
<<<<<<< HEAD
\protect\hypertarget{section-11}{}
=======
\protect\hypertarget{section-9}{}
>>>>>>> e735aca38ee9621f00e78475fcc0664f62caa5b7

\textbf{EXPERMIENTO BINOMIAL}

Es un experimento que debe cumplir las siguientes condiciones:

\textbf{1.} El experimento consta de una secuencia de \textbf{n} ensayos
idénticos.

\textbf{2.} En cada ensayo hay dos resultados posibles. A uno de ellos
se le llama \textbf{éxito} y al otro, \textbf{fracaso}.

\textbf{3.} La probabilidad de éxito es constante de un ensayo a otro,
nunca cambia y se denota por \textbf{p}. Por ello, la probabilidad de
fracaso será \textbf{1-p}.

\textbf{4.} Los ensayos son \textbf{independientes}, de modo que el
resultado de cualquiera de ellos \textbf{\emph{no}} influye en el
resultado de cualquier otro ensayo.

\end{frame}

\begin{frame}{}
<<<<<<< HEAD
\protect\hypertarget{section-12}{}

\textbf{FUNCIÓN DE PROBABILIDAD BINOMIAL}

Para un experimento binomial,sea \textbf{p} la probabilidad de un
``éxito'' y \textbf{1-p} la probabilidad de un ``fracaso'' en un solo
ensayo, entonces la probabilidad de obtener \textbf{x} éxitos en
\textbf{n} ensayos, está dado por la función de probabilidad
\textbf{\emph{f(x)}}:

\includegraphics{~/GitHub/DiplomadoR_Acuicultura_v21/Imagenes/Clase5/FPbinomial.png}
=======
\protect\hypertarget{section-10}{}

\textbf{VARIABLE ALEATORIA DISCRETA CON DISTRIBUCIÓN BINOMIAL NEGATIVA}

\includegraphics{~/GitHub/DiplomadoR_Acuicultura_v21/Imagenes/Clase5/Binega.png}
>>>>>>> e735aca38ee9621f00e78475fcc0664f62caa5b7

\end{frame}

\begin{frame}{}
<<<<<<< HEAD
\protect\hypertarget{section-13}{}

\textbf{VARIABLE ALEATORIA DISCRETA CON DISTRIBUCIÓN BERNOULLI}

VENENO PARALIZANTE DE LOS MARISCOS (VPM)

Producido en Chile por una microalga llamada Alexandrium catenella.

\includegraphics{~/GitHub/DiplomadoR_Acuicultura_v21/Imagenes/Clase5/LetalidadVPM.png}

Intoxicaciones por VPM en Chile 1972-2002 456 enfermos, 30 fallecidos
(total = 486) Letalidad= 30 / 486 = 0.0617
=======
\protect\hypertarget{section-11}{}

\textbf{SIMULAR VARIABLE ALEATORIA DISCRETA CON DISTRIBUCIÓN BINOMIAL
NEGATIVA}

La abundancia de parásitos como el piojo de mar es una variable distreta
con distribución binomial negativa, esto significa que hay muchos peces
con pocos parásitos\\
(ej= 0 o 1) y poco con muchos parásitos.

\includegraphics{~/GitHub/DiplomadoR_Acuicultura_v21/Imagenes/Clase5/parasitos.png}
>>>>>>> e735aca38ee9621f00e78475fcc0664f62caa5b7

\end{frame}

\begin{frame}{}
<<<<<<< HEAD
\protect\hypertarget{section-14}{}

\textbf{DESCRIBIR EL COMPORTAMIENTO DE MORTALIDAD POR VPM}

Ejemplo: Ocurre un evento de marea roja no detectado con A. catenella,
80 personas resultan intoxicadas en todo Chile.

Si la probabilidad de muerte x VPM es de 0,0617.

¿Cuántas personas morirán?

\includegraphics{~/GitHub/DiplomadoR_Acuicultura_v21/Imagenes/Clase5/LetalidadVPM.png}
=======
\protect\hypertarget{section-12}{}

\textbf{HISTOGRAMA Y BOXPLOT}

\includegraphics{~/GitHub/DiplomadoR_Acuicultura_v21/Imagenes/Clase5/caligulus.png}
>>>>>>> e735aca38ee9621f00e78475fcc0664f62caa5b7

\end{frame}

\begin{frame}{}
<<<<<<< HEAD
\protect\hypertarget{section-15}{}
=======
\protect\hypertarget{section-13}{}
>>>>>>> e735aca38ee9621f00e78475fcc0664f62caa5b7

\textbf{Práctica con Rmardown}

\textbf{\emph{Clase 5 -- Variables discretas}}

\end{frame}

\begin{frame}{}
<<<<<<< HEAD
\protect\hypertarget{section-16}{}
=======
\protect\hypertarget{section-14}{}
>>>>>>> e735aca38ee9621f00e78475fcc0664f62caa5b7

\textbf{TRABAJO EN SALAS}

\textbf{1).} \textbf{Guía de trabajo programación con R disponible en
drive.}

\textbf{2).} \textbf{La tarea se realiza en Rstudio.cloud}.

Ingresa al siguiente proyecto de
\textbf{\href{https://rstudio.cloud/spaces/135178/project/2447826/}{Rstudio.Cloud}}

\end{frame}

\begin{frame}{}
<<<<<<< HEAD
\protect\hypertarget{section-17}{}
=======
\protect\hypertarget{section-15}{}
>>>>>>> e735aca38ee9621f00e78475fcc0664f62caa5b7

\textbf{RESUMEN DE LA CLASE}

\begin{itemize}
\item
  Revisar ventajas de elaborar reportes dinámicos con
  \textbf{\emph{Rmardown}}.
\item
  Escribir un código de programación con \textbf{\emph{Rmardown}}
\item
  Elaborar diferentes reportes dinámicos.
\end{itemize}

\end{frame}

\end{document}
